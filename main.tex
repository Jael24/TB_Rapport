% Preamble
% ---
\documentclass[paper=a4, fontsize=11pt]{scrartcl}

% Packages
% ---
\usepackage{geometry}
\geometry{
  a4paper,
  left=15mm,
  right=15mm,
  headheight=50mm,
  top=30mm,
  bottom=16mm,
  footskip=10mm
}
\usepackage{graphicx}
\usepackage[utf8]{inputenc}     % UTF-8 support
\usepackage{amsmath,amsfonts}   % Advanced math typesetting
\usepackage{gauss}              % Matrices reduction operations
\usepackage{lastpage}           % Reference to last page
\usepackage[parfill]{parskip}   % No indent at start of new line
\usepackage{hyperref}           % Table of contents with clickable links
\usepackage{subfig}             % Figure repartition on page
\usepackage{ragged2e}
\usepackage[T1]{fontenc}
\usepackage[french]{babel}
\usepackage{listings}           % Source code formatting and highlighting
\frenchbsetup{StandardLists=true}
\hypersetup{
    colorlinks,
    citecolor=black,
    filecolor=black,
    linkcolor=black,
    urlcolor=black
}
\usepackage{fancyhdr}           % Create headers and footers
\pagestyle{fancy}

% Header and footer things
% ---
% Header of all pages
\lhead{\includegraphics[width=5cm]{img/logo.png}}
\rhead{}

% Footer of all pages
\renewcommand{\footrulewidth}{0.4pt}
\lfoot{Système de logging multi-niveau}
\cfoot{}
\rfoot{Page \textbf{\thepage} \ sur \textbf{\pageref{LastPage}}}

% Header and footer of the first page
\fancypagestyle{firstpage}{
  \renewcommand{\headrulewidth}{0pt}
  \lhead{\includegraphics[width=5cm]{img/logo.png}}
  \rhead{Département : Technologies de l'information et de la communication\linebreak Filière : Informatique et systèmes de communication\linebreak Orientation : Informatique logicielle}
  \renewcommand{\footrulewidth}{0pt}
  \lfoot{}
  \rfoot{}
}

% Document
% ---
\begin{document}

% Don't number sections
% \setcounter{secnumdepth}{0}

% Title page ========================================================
\begin{titlepage}
  \thispagestyle{firstpage}
  \begin{center}
    \vspace*{5cm}
    
    \Huge
    \textbf{Travail de Bachelor}

    \vspace{1.5cm}
    \LARGE
    Analyse et implémentation d'un système de logging multi-niveau pour une plateforme Smart Grid
  \end{center}

  \vspace{6cm}
  \begin{tabbing}
    \linespread{3}\textbf{Étudiant :} \hspace{12em} \= Jael Dubey\\\\

    \textbf{Travail proposé par :} \> Jonathan Bischof\\
    \> DEPsys SA\\
    \> Route du Verney 20B\\
    \> 1070 Puidoux\\\\

    \textbf{Enseignant responsable :} \> Nastaran Fatemi\\\\

    \textbf{Année académique :} \> 2019-2020
  \end{tabbing}

  \vspace{3cm}
  \begin{flushright}
    Yverdon-les-Bains, le \textit{JOUR} \textit{MOIS} 2020
  \end{flushright}
\end{titlepage}

\newpage
% Contents page =====================================================
\renewcommand{\contentsname}{Table des matières}
\tableofcontents

\newpage
% Document start ====================================================
\section{Introduction}
\section{Évaluation}

\subsection{Critères d'évaluation}

Pour réaliser une évaluation de différents système de gestion de log, il faut obligatoirement choisir des critères d'évaluation. Ces critères se basent sur deux sources. La première étant les demandes formulées par DEPsys, et la deuxième provient des différentes fonctionnalités nécessaires à un système de gestion de logs. Voici les critères retenus :

\begin{itemize}\itemsep0pt \parskip0pt \parsep0pt
\item La collecte des logs
\subitem Approche minimaliste ou maximaliste, est-ce que le système installe un agent sur le dispositif émetteur de log qui lui envoie uniquement les informations les plus importantes (approche minimaliste, méthode PUSH), ou est-ce que le système reçoit tous les logs et les enregistre tous (approche maximaliste, méthode PULL) ?\\
\item L'agrégation centralisée des logs
\subitem L'agrégation des logs est un défi, car après avoir collecté les logs, il faut tous les regrouper dans un même endroit, alors qu'ils peuvent avoir des formats différents. De plus, ils peuvent être généré très rapidement (s'exprime en EPS, Event Per Second), il faut donc être capable de traiter et regrouper ces logs de manière efficace.\\
\item Le stockage à long terme et la durée de rétention des logs
\subitem Après avoir agrégé ces informations, il faut maintenant faire des choix quant à leur stockage. L'idéal serait de garder tous les logs indéfiniment, mais chaque information stockée à un coût. Il faut donc avoir une stratégie de rétention qui permette de supprimer ou garder tel type de log.\\
\item La rotation des fichiers de logs
\subitem La rotation consiste à rendre automatique la stratégie de rétention et/ou de stockage des logs.\\
\item L'analyse des logs (en temps réel et en vrac après une période de stockage)
\subitem L'analyse des logs est, en quelque sorte, le but de tout le système de gestion de logs. En effet, il ne sert à rien de stocker de l'information sur un système si l'on en fait rien. L'analyse est donc la pour synthétiser les informations contenues dans les logs.\\
\item Les rapports
\subitem Il doit être possible pour un système de gestion des logs d'effectuer des recherches sur les informations stockées et de rédiger des rapports.\\
\item Visionnage et gestion des alertes
\subitem Un des buts d'un système de gestion de logs est de pouvoir réagir très vite à un problème, voire même de l'anticiper. Ceci passe par une émission d'alerte et d'une possibilité de visionnage des données, si possible en temps réel.\\
\item Popularité
\subitem Voici un critère qui change en permanence, mais qui a son importance lors d'un choix d'outil informatique. En effet, un logiciel sur le déclin sera de plus en plus dur à supporter, alors qu'un outil trop jeune n'a souvent que trop peu d'utilisateurs qui pourraient partager leurs connaissances sur les forums. L'idéal étant donc un logiciel populaire et qui est en pleine croissance.
\end{itemize}

En plus de ces critères, les coûts d'utilisation seront évalués.\\

\subsection{Choix des différents systèmes à évaluer}

Comme pour le choix des critères d'évaluation, les différents systèmes qui vont être analysés ont été définis soit par DEPsys, soit par des recherches dans les nombreux classement de \og Log Management Tools \fg disponible sur internet.\\
Quatre classements différents ont été sélectionnés afin d'avoir plusieurs avis différents, tout en restant dans une quantité raisonnable d'analyses. Les classements qui allaient être utilisés devaient être neutre. On entend par là que le site réalisant le classement ne doit pas proposer, par exemple, une solution cloud utilisant un certain outil de gestion de log, auquel cas son classement serait forcément biaisé. Il fallait également que le classement soi récent, étant donné la vitesse d'évolution générale des outils informatiques. Une période d'un an maximum a été définie.\\\\
Les classements suivants ont été choisis :
\begin{itemize}
    \item \href{https://www.softwaretestinghelp.com/log-management-software/}{Top 8 BEST Log Management Software}
    \subitem Classement datant de mars 2020, et publié par \textit{SoftwareTestingHelp}.
    \item \href{https://www.ittsystems.com/log-manager-software-and-tools/}{Best Log Manager \& Monitoring Software \& Tools}
    \subitem Classement datant de janvier 2020, et publié par \textit{iTT Systems}.
    \item \href{https://www.addictivetips.com/net-admin/linux-log-management-tools/}{6 Best Log Management Tools}
    \subitem Classement datant de août 2019, et publié par \textit{AddictiveTips}.
    \item \href{https://www.comparitech.com/net-admin/log-management-tools/}{13 Best Log Management \& Analysis Tools}
    \subitem Classement datant de août 2019, et publié par \textit{Comparitech}.
\end{itemize}

Les quatre sites internet retenus sont de simples portails informatiques, publiant des articles sur divers sujets informatiques. Lorsque l'on effectue une recherche Google afin de trouver des classements de système de gestion de logs, on tombe régulièrement sur des articles de \textit{DNSStuff} et \textit{Sematext}. Ceux-ci n'ont pas été utilisé car le premier appartient à SolarWinds, et le second propose des solutions cloud utilisant des outils de gestion de logs. Ils ont donc été jugés non-neutre.

Afin de réaliser un classement regroupant le contenu de ces 4 tops, une note a été attribuée à chaque position de chaque classement. Deux critères ont été intégré dans le calcul de la note : la position (mieux on est positionné dans son propre classement, plus on aura de points), ainsi que le nombre total d'outils cités. Il est en effet plus difficile d'être bien classé dans un classement contenant 5 outils que dans un classement en contenant 10. Pour finir, afin de faciliter la lecture, la note attribuée se situe entre 0 (le plus mauvais), et 100 (la meilleure note possible). La formule suivante a donc été appliquée :
\[\frac{i * 100}{n}\]
i étant la position dans le classement, et n le nombre total d'outil dans le classement.\\

Voici les 4 classements et les points obtenus par chaque système :\\
    \begin{tabular}{ |p{6cm}|p{1cm}|  }
     \hline
     \multicolumn{2}{|c|}{SoftwareTestingHelp} \\
     \hline
     Système & Points\\
     \hline
     SolarWinds Log Analyzer & 100\\
     Sematext Logs & 89\\
     Splunk & 78\\
     ManageEngine EventLog Analyzer & 67\\
     LogDNA & 56\\
     Fluentd & 44\\
     Logalyze & 33\\
     Graylog & 22\\
     Netwrix Auditor & 11\\
     \hline
    \end{tabular}
\hfill
    \begin{tabular}{ |p{6cm}|p{1cm}|  }
     \hline
     \multicolumn{2}{|c|}{Comparitech} \\
     \hline
     Système & Points\\
     \hline
     ManageEngine EventLog Analyzer & 100\\
     SolarWinds Papertrail & 92\\
     Loggly & 83\\
     PRTG Network Monitor & 75\\
     Splunk & 67\\
     Fluentd & 58\\
     Logstash & 50\\
     Kibana & 43\\
     Graylog & 33\\
     XpoLog & 25\\
     ManageEngine SyslogForwarder & 17\\
     TekWire Managelogs & 8\\
     \hline
    \end{tabular}

    \begin{tabular}{ |p{6cm}|p{1cm}|  }
     \hline
     \multicolumn{2}{|c|}{AddictiveTips} \\
     \hline
     Système & Points\\
     \hline
     SolarWinds Log \& Event Manager & 100\\
     PRTG Network Monitor & 83\\
     Lepide & 67\\
     McAfee Enterprise Log Manager & 50\\
     Veriato & 33\\
     Splunk & 17\\
     \hline
    \end{tabular}
\hfill
    \begin{tabular}{ |p{6cm}|p{1cm}|  }
     \hline
     \multicolumn{2}{|c|}{iTT Systems} \\
     \hline
     Système & Points\\
     \hline
     Solarwinds Log \& Event Manager & 100\\
     PRTG Network Monitor & 83\\
     Lepide & 67\\
     McAfee Enterprise Log Manager & 50\\
     Veriato & 33\\
     Splunk & 17\\
     \hline
    \end{tabular}

Et voici le classement global, après addition des points obtenus dans les différents classements :

\centering
\begin{tabular}{ |p{2cm}|p{6cm}|p{1cm}| } 
    \hline
     & Système & Points\\
    \hline
    1 & Splunk & 229\\
    2 & SolarWinds Papertrail & 192\\
    3 & ManageEngine EventLog Analyzer & 184\\
    4 & SolarWinds Loggly & 166\\
    5 & PRTG Network Monitor & 158\\
    6 & Fluentd & 102\\
    7 & SolarWinds Log Analyzer & 100\\
    8 & SolarWinds Log \& Event Manager & 100\\
    9 & ELK Stack (Kibana + Logstash)  & 92\\
    10 & Sematext Logs & 89\\
    11 & Graylog & 88\\
    12 & Lepide & 67\\
    13 & LogDNA & 56\\
    14 & Nagios Log Server & 50\\
    15 & McAfee Enterprise Log Manager & 50\\
    16 & Logalyze & 33\\
    17 & Veriato & 33\\
    18 & XpoLog & 25\\
    19 & ManageEngine Syslog Forwarder & 18\\
    20 & Netwrix Auditor & 11\\
    21 & TekWire Managelogs & 8\\
    \hline
\end{tabular}

\justify

Avec ces résultats, on constate qu'il y a un certain nombre de systèmes appartenant à SolarWinds. Après une étude légèrement plus approfondie sur ces différents systèmes, il a été décidé de garder les suivants pour l'évaluation :

\begin{enumerate}
    \item Elastic Stack
    \subitem Choisi par DEPsys, probablement la plus populaire.
    \item Splunk
    \subitem 1\up{er} du classement.
    \item SolarWinds Loggly
    \subitem De tous les outils appartenant à SolarWinds, je voulais n'en choisir qu'un. Loggly me paraissaît le plus approprié au cas d'utilisation de ce travail de Bachelor.
    \item Graylog
    \subitem Suggéré par DEPsys, et semble avoir une bonne documentation.
\end{enumerate}

\subsection{Évaluation}
\subsubsection{Elastic Stack}
La \og Elastic Stack \fg, ou \og Suite Elastic \fg en français, anciennement appelée \og ELK Stack \fg est composée de plusieurs outils :
\begin{itemize}
    \item Elasticsearch
    \subitem Un moteur de recherche RESTful.
    \item Kibana
    \subitem Un outil de visualisation.
    \item Logstash
    \subitem Un pipeline d'ingestion de log.
    \item Beat
    \subitem Une famille d'agent dédié au transfert de données.
\end{itemize}

\centering
\begin{tabular}{ |p{4cm}||p{12cm}|  }
    \hline
    \multicolumn{2}{|c|}{Elastic Stack} \\
    \hline
    Collecte & La collecte des logs se fait en approche minimaliste. La suite Elastic contient l'outil Beat, qui est donc une famille d'agent. On installe un agent beat (p. ex. FileBeat, MetricBeat, etc.) sur le système générant les logs, et cet agent envoie les données vers le le serveur.\\
    \hline
    Agrégation centralisée & Se fait via Logstash. Peux supporter beaucoup d'événements par seconde (> 10'000 EPS). Compatible avec énormément de type de logs. Permet d'analyser et transformer les logs en temps réel. Logstash dispose d'une API permettant de créer nos propres plug-in, si les sources de données ne sont pas compatible nativement.\\
    \hline
    Stockage et rétention & Le stockage se fait avec Elasticsearch. Il n'y a pas de rétention des données de bases avec Elasticsearch. Il est cependant possible de le faire avec Elastic-Curator, qui est un outil permettant de gérer un cluster Elasticsearch.\\
    \hline
    Rotation & La rotation se fait avec Elastic-Curator\\
    \hline
    Analyse & Elasticsearch et ses requêtes poussées permettent de faire des recherches avancées.\\
    \hline
    Rapport & Les rapports peuvent être générés depuis Kibana.\\
    \hline
    Visionnage et alertes & La visualisation des données en temps réels peut se faire avec Kibana. La gestion des alertes se fait également via Kibana. Il est possible de paramétrer des alertes classiques, qui se déclenchent suivant des règles précises. Et il est également possible de paramétrer des alertes suivant un algorithme d'apprentissage automatique, qui détectera des événements inhabituels.\\
    \hline
    Popularité & La suite Elastic est très certainement la plus populaire actuellement. Elle bénéficie d'une grande communauté active. Au niveau de la tendance, on peut voir une grande croissance entre les années 2016 et 2019. Ces derniers mois, cela semble se stabiliser.\\
    \hline
    Coûts &  La suite Elastic propose différents abonnements. Il y a une offre gratuite, mais celle-ci ne contient pas de gestion d'alerte et de création de rapport. Les prix pour les offres payantes ne sont pas publics. Il faut contacter Elastic et les prix varient en fonction de la taille du système à implémenter.\\
    \hline
\end{tabular}
\justify

\subsubsection{Graylog}
Graylog un outil de gestion de logs. Il dispose de deux versions : Open Source et Enterprise.

\centering
\begin{tabular}{ |p{4cm}||p{12cm}|  }
    \hline
    \multicolumn{2}{|c|}{Graylog} \\
    \hline
    Collecte & La collecte des logs se fait en approche minimaliste. Graylog possède un outil appelé \og Sidecar \fg qui permet de gérer plusieurs type d'agent, y compris l'outil Beat de la suite Elastic.\\
    \hline
    Agrégation centralisée & Graylog permet de gérer \og d'énormes \fg jeux de donnée et de les traiter selon des règles définies par l'utilisateur. En plus des règles classiques, comme la géographie, le type, etc., Graylog permet de faire des listes noires de logs.\\
    \hline
    Stockage et rétention & Le stockage se fait avec MongoDB et Elasticsearch. Graylog offre une solution (Graylog Archive) de rétention des données, disponible avec la version Enterprise, qui peut être paramétrée.\\
    \hline
    Rotation & La rotation se fait avec Graylog Archive.\\
    \hline
    Analyse & Graylog utilisant Elasticsearch, il dispose de ses requêtes poussées permettant de faire des recherches avancées. Graylog possède également son langage de requête, basé sur Apache Lucene.\\
    \hline
    Rapport & La création de rapport est une fonctionnalité de Graylog Enterprise. Il est possible de les configurer depuis l'interface de Graylog.\\
    \hline
    Visionnage et alertes & La visualisation des données en temps réels se fait avec l'interface de Graylog. La gestion des alertes est incluse dans le Graylog de base. Elle permet de définir des alertes selon des règles. Graylog possède également un \og Store \fg proposant, entre autre, des fonctionnalités liées aux alertes, développées par la communauté.\\
    \hline
    Popularité & Graylog n'est pas arrivé très haut de manière générale dans les tops, mais il est en revanche souvent cité. La courbe de tendance de Graylog est en croissance régulière depuis 2008.\\
    \hline
    Coûts &  Graylog propose deux versions : \og Open Source \fg et \og Enterprise \fg. La première est gratuite mais propose quelques fonctionnalités en moins, comme les rapports programmés ou le support technique. Les coûts de la version \og Enterprise \fg ne sont pas disponible, il faut contacter Graylog. À noter que la version \og Enterprise \fg est gratuite jusqu'à une utilisation de 5 GB par jour.\\
    \hline
\end{tabular}
\justify

\subsubsection{SolarWinds Loggly}
SolarWinds Loggly un outil de gestion de logs SaaS (Software-as-a-Service, dans le cloud). Il dispose de plusieurs version, dont une gratuite.

\centering
\begin{tabular}{ |p{4cm}||p{12cm}|  }
    \hline
    \multicolumn{2}{|c|}{SolarWinds Loggly} \\
    \hline
    Collecte & L'envoi de données à Loggly est relativement simple. La seule contrainte est qu'il s'agisse de texte. Il n'y a pas besoin d'avoir d'agent (envoi de log via un endpoint), mais il est également possible d'en utiliser.\\
    \hline
    Agrégation centralisée & \\
    \hline
    Stockage et rétention & La rétention des données est plutôt courte (7 jours dans la version gratuite), et ensuite, les logs peuvent être sauvegardés dans une instance S3 de AWS.\\
    \hline
    Rotation & La rétention se fait automatiquement après un certain nombre de jour.\\
    \hline
    Analyse & Loggly possède son propre langage de requête, qui est basé sur Apache Lucene. Il est également possible d'analyser les logs en temps réel via l'interface de Loggly.\\
    \hline
    Rapport & Il est possible de réaliser des rapport dans plusieurs format depuis l'interface graphique.\\
    \hline
    Visionnage et alertes & Il est possible de configurer des alertes selon des règles classiques, et également sur des événements inhabituels. Le visionnage des données en direct se fait via l'interface graphique de Loggly.\\
    \hline
    Popularité & Loggly était en croissance entre 2008 et 2016, mais depuis cette année-ci, sa courbe de tendance est en décroissance.\\
    \hline
    Coûts & Loggly propose une version gratuite et trois versions payantes. La version gratuite est limitée à un volume de 200 MB par jour, un seul utilisateur pouvant se connecter sur une instance, et ne contient pas certaines fonctionnalités comme la gestion des alertes. Les versions payantes varient entre 79 et 279 USD par mois, et proposent chacune quelques fonctionnalités en plus, et un volume de moins en moins limité.\\
    \hline
\end{tabular}
\justify

\subsubsection{Splunk}
Splunk est un outil de gestion de logs. Il est séparé en trois \og parties \fg, chacune étant responsable de plusieurs choses :

\begin{itemize}
    \item Universal Forwarder
    \subitem Effectue la collecte et envoie les données à l'indexer.
    \item Indexer
    \subitem Effectue le stockage des logs.
    \item Search Head
    \subitem Permet de lire dans l'index.
\end{itemize}
Chacune de ces fonctions peut être transformé en cluster si de grand volume de données sont à traiter.
Splunk peut être disponible en version \og sur site \fg ou \og cloud \fg.

\centering
\begin{tabular}{ |p{4cm}||p{12cm}|  }
    \hline
    \multicolumn{2}{|c|}{Splunk} \\
    \hline
    Collecte & La collecte des logs se fait en approche minimaliste. Ceci via les \og Universal Forwarder \fg qui sont des agents à installer sur le système générant les logs. Il envoie ensuite les données vers l'indexer.\\
    \hline
    Agrégation centralisée & \\
    \hline
    Stockage et rétention & Le stockage des logs se fait avec l'indexer. Concernant la rétention des données, Splunk la gère avec des \og buckets \fg. Concrètement, un bucket possède une durée qui détermine le temps qu'une donnée va passer dedans. P. ex., on peut avoir un bucket de 30 jours où les logs seront accessible et analysable librement. Puis, passé ces 30 jours, ils iront dans un autre bucket où ils seront compressé pour le stockage à long terme.\\
    \hline
    Rotation & Se fait via les buckets.\\
    \hline
    Analyse & L'analyse est possible via l'interface de Splunk. Celle-ci permet de trier et filtrer les logs selon de nombreux critères.\\
    \hline
    Rapport & La génération de rapport est possible depuis l'interface de Splunk.\\
    \hline
    Visionnage et alertes & Le visionnage et la gestion des alertes est également possible depuis l'interface de Splunk. Pour les alertes, elles sont définissables selon des règles classiques (p. ex. logs provenant d'une adresse IP particulière, etc.).\\
    \hline
    Popularité & D'après les courbes de Google Trends, le système Splunk est en constante croissance depuis 2010. Dans le classement des différents tops consultés, Splunk arrive en bonne position. Splunk bénéficie d'une documentation relativement grande et d'un forum.\\
    \hline
    Coûts & Les coûts sont en \og infrastructure-based pricing \fg et ne sont donc pas fixe. Mais Splunk Enterprise commence à 150\$ par mois. Splunk ne propose pas de version gratuite (mise à part l'essai de 60 jours).\\
    \hline
\end{tabular}
\justify

\subsubsection{Comparaison}

Plusieurs tests ont été effectués afin de pouvoir effectuer une comparaison de ces systèmes. Comme ces différents tests obligent une prise en main concrète des logiciels, ceux-ci ont également permis d'avoir une idée de la facilité d'utilisation, de la flexibilité ainsi que de l'utilisabilité ( \og user-friendliness \fg) des systèmes.

Voici les versions des logicielles utilisées pour effectuer ces tests :

\begin{itemize}
    \item Elastic Stack
    \subitem Elasticsearch 7.6.2
    \subitem Filebeat 7.6.2
    \subitem Logstash 7.6.2
    \subitem Kibana 7.6.2
    \item Graylog
    \subitem Graylog 3.2.4-1 (Virtual Appliance)
    \subitem Filebeat 7.6.2
    \item Splunk
    \subitem Splunk 8.0.2.1
    \subitem Splunk Universal Forwarder 8.0.2.1
    \item Loggly
    \subitem Loggly Lite (Cloud)
\end{itemize}

\textbf{Test de débit d'ingestion de log}

Le premier test de cette comparaison consiste en un test de débit. Cela permet d'avoir une première idée de la performance des applications.

Conditions de tests :
\begin{itemize}
    \item Chargement d'un fichier de logs.
    \subitem Le fichier contient 98'305 logs identiques.
    \subitem Log : 2020-02-27 09:06:24.596 INFO  o.s.s.c.ThreadPoolTaskScheduler -> Shutting down ExecutorService 'taskScheduler'
    \item On effectue un test d'ingestion simple (on stocke le log brut).
    \item On effectue un test d'ingestion avec filtre (on parse les différentes parties du log).
    \subitem Date et heure : 2020-02-27 09:06:24.596
    \subitem Niveau du log : INFO
    \subitem Classe Java : o.s.s.c.ThreadPoolTaskScheduler
    \subitem Message : Shutting down ExecutorService 'taskScheduler'
    \item Le résultat est un débit exprimé en EPS (Event Per Second), qui correspond aux nombres de logs que le système aura pu ingéré en une seconde.
\end{itemize}

Résultats :

\centering
\begin{tabular}{ |p{3cm}|p{3cm}|p{3cm}|p{3cm}|p{3cm}|  }
    \hline
    & Elastic Stack & Graylog & Splunk & Loggly \\
    \hline
    Sans filtrage & 4'898 & 14'044 & - & 46'295 (\textasteriskcentered) \\
    \hline
    Avec filtrage & 9'181 & 7'868 & 2'628 & -\\
    \hline
\end{tabular}
\justify

(*) Loggly n'acceptant que des fichiers de taille inférieure à 5 MB, le fichier à été réduit à 3'999 logs.

Splunk favorise l'extraction d'informations (parsing) après avoir stocké les logs dans le système. Il est donc compliqué d'ajouter un filtrage avant le stockage et ça n'a donc pas été réalisé. Par contre, Splunk effectue automatiquement une extraction de la date des logs. Son débit à donc été classé dans la catégorie \og avec filtrage \fg.
Loggly ayant été écarté assez tôt de l'évaluation, la partie \og avec filtrage \fg n'a pas été testée. \\

Conclusion : Sans filtrage, on constate que Graylog possède un débit supérieur à Elastic Stack. Loggly n'acceptant pas des fichiers de grande taille, il est difficilement comparable. Avec filtrage, étonnemment, la Suite Elastic gagne en débit. Elle possède donc un meilleur débit que Graylog et Splunk.
Après ce test, outre les résultats comptables, il se dessine surtout une tendance vers Graylog et Elastic Stack, car ces deux systèmes sont plus \og ouvert \fg que les autres. Il n'y a pas eu de difficultés particulières lors de l'utilisation, pas de contraintes comme la taille du fichier ou encore le moment du filtrage au sein du pipeline.

\textbf{Test d'ingestion d'une grande quantité de logs}

Ce test a pour but de mettre en évidence les limites de certains systèmes lors de l'ingestion de gros fichier de logs. Il permet également d'avoir une brève vue sur la constance du débit mesuré lors du test précédent.

Conditions de tests :
\begin{itemize}
    \item Chargement d'un fichier de logs.
    \subitem Le fichier contient 999'999 logs identiques.
    \subitem Log : 2020-02-27 09:06:24.596 INFO o.s.s.c.ThreadPoolTaskScheduler -> Shutting down ExecutorService ’taskScheduler’
    \item On n'effectue aucun traitement sur le log.
    \item Le résultat est soit une limite supérieure, soit \og > 1'000'000 \fg
\end{itemize}

Résultats : 

\centering
\begin{tabular}{ |p{3cm}|p{3cm}|p{3cm}|p{3cm}|p{3cm}|  }
    \hline
    Elastic Stack & Graylog & Splunk & Loggly \\
    \hline
    > 1'000'000 & > 1'000'000 & > 1'000'000 & \textasciitilde 4'000 (5 MB) \\
    \hline
\end{tabular}
\justify

Conclusion :
Ce test permet simplement de montrer que les systèmes travaillant avec un agent, soit selon la méthode PUSH, n'ont pas de problème de taille de fichier, car l'agent lit et envoie les logs au fur et à mesure. Loggly est le seul ici utilisant la méthode PULL. Au niveau des débits, ceux d'Elastic Stack et Graylog ont été divisé par 2, alors que celui de Splunk est resté stable. Pour Loggly, il n'y a pas eu de différence, étant donné qu'il est limité en taille de fichier.


\textbf{Test de consommation CPU}

Ce test-ci a pour but de pouvoir comparer les systèmes de gestion de logs Elastic Stack et Graylog en terme de performance.

Conditions de test :
\begin{itemize}
    \item Chargement du même fichier de log que pour le test de débit.
    \item L'utilisation du CPU sera réduite au minimum hors système à tester.
    \item Ordinateur de test : ASUS UX360UAK
    \item Processeur : Intel Core i7-7500U CPU @ 2.70GHz x 4
    \item Lors de la réception, l'agent Filebeat se situera sur une autre machine.
\end{itemize}

Résultats :

Elastic Stack (Logstash - Elasticsearch - Kibana), réception des logs :\\
\includegraphics[width=19cm]{img/screenshots/Elastic_CPU_MEM_Receive_modified.png}
Graylog, réception des logs :\\
\includegraphics[width=19cm]{img/screenshots/Graylog_CPU_MEM_Receive_modified.png}
Filebeat, envoi des logs :\\
\includegraphics[width=19cm]{img/screenshots/Filebeat_send_modified.png}



\appendix

\section{Images}

\section{Détails des tests}

\subsection{Tests de débit d'ingestion}

Graylog (3.2.4 Virtual appliance) avec Filebeat :

Sans extractor (module de Graylog permettant le parsing des logs) :

Début du chargement : 10:09:17 \\
Fin du chargement : 10:09:24 \\
Temps de chargement : 7 secondes \\
Débit moyen : \textasciitilde 14'043,47 EPS \\

Avec extractor :

Début du chargement : 07:46:47.153 \\
Fin du chargement : 07:46:59.648 \\
Temps de chargement : 12.495 secondes \\
Débit moyen : \textasciitilde 7'867,55 EPS \\

Elastic Stack sans Logstash :

Début du chargement : 10:34:34.200 \\
Fin du chargement : 10:35:03.622 \\
Temps de chargement : 29.422 secondes \\
Débit moyen : \textasciitilde 3'341,21 EPS \\

Elastic Stack complet (sans filtrage du log) :

Début du chargement : 10:44:11.851 \\
Fin du chargement : 10:44:31.920 \\
Temps de chargement : 20.069 secondes \\
Débit moyen : \textasciitilde 4'898,35 EPS \\

Elastic Stack complet (avec filtrage du log) :

Début du chargement : 10:26:37.308 \\
Fin du chargement : 10:26:48.015 \\
Temps de chargement : 10.707 secondes \\
Débit moyen : \textasciitilde 9'181,38 EPS \\

Splunk :

Avec filtrage : \\
Donné (metrics.log de Splunk) : 2'628 EPS \\

Solarwinds Loggly avec endpoint bulk :
Avec Loggly, il est impossible de charger un fichier de plus de 5 MB. Il a donc été réduit à 39'999 lignes.

Début du chargement : 08:42:11.360 \\
Fin du chargement : 18:42:12.224 \\
Temps de chargement : 0.864 seconde \\
Débit moyen : \textasciitilde 46'295,14 EPS \\

\section{Journal de travail}


\subsection{Jeudi 13 février}
    Première réunion avec Nastaran, Jonathan et Pascal. Jonathan et Pascal ont expliqué leur vision du TB à travers une présentation, puis nous avons planifié le travail de Bachelor. Notamment les dates de fin d'évaluation (avec la présentation à DEPsys), et de fin de développement du use-case.
\subsection{Mercredi 19 février}
    Début du Travail de Bachelor. J'ai commencé par suivre un tuto afin de maîtriser les bases du langage \LaTeX, ce qui me sera utile pour tout ce qui est rédactionnel. Ensuite, j'ai commencé à revoir la présentation de Pascal afin de bien comprendre (notamment les technologies que je ne connais pas).
\subsection{Jeudi 20 février}
    J'ai regardé plusieurs vidéos qui présentent les différentes technologies que je dois évaluer.
\subsection{Mercredi 26 février}
    J'ai décidé de commencer à évaluer plus en profondeur Elasticsearch en premier, car Prometheus à comme contrainte de ne pas gérer les logs textuels, mais uniquement des métriques numériques. Cependant, d'après plusieurs lectures, je pense qu'il pourrait être intéressant de mixer les deux solutions. J'ai donc installé les outils de la suite ELK, et suivi des tutos plus concret en ce qui concerne Elasticsearch (insertion de donnée, recherches, etc.).
\subsection{Jeudi 27 février}
    Deuxième réunion avec Nastaran. Elle me propose de recentrer mes recherches sur la partie \og Log Analysis \fg, donc rechercher directement l'intégration de l'analyse de logs avec ELK par exemple. Après la réunion, j'ai donc continuer mes recherches dans ce sens et ai suivi la vidéo de elastic qui concerne l'analyse de logs.
\subsection{Mercredi 04 mars}
    J'ai commencé à écrire ce journal afin de mieux me rappeler de ce que j'ai fais, ainsi que d'être plus structuré. Je commence également à utiliser Zotero, qui permet d'enregistrer tous les liens que je trouve intéressant, ainsi que de créer une bibliographie. J'ai également décidé de m'intéresser à Graylog en plus de ELK.
\subsection{Jeudi 05 mars}
    J'ai exploré plus en profondeur les articles de type \og Elastic Stack versus Graylog \fg, et je vais donc inclure la stack \og Graylog server, MongoDB et Elasticsearch \fg dans le comparatif. Cette suite-là me semble très appropriée au traitement et à l'analyse de logs.
    J'ai été à la réunion avec Nastaran à 11h30. Suite à cette réunion, nous avons décidé qu'il fallait que je fasse une synthèse des mes recherches et que je la présente en quelques slides le jeudi 12 mars.
\subsection{Lundi 09 mars}
    J'ai commencé à faire la synthèse de mes recherches. Je vais donc la faire en 3 étapes :
    \begin{enumerate}
    \item Choix des critères d'évaluations
        \begin{enumerate}
            \item Selon des recherches au sujet des caractéristiques d'un \og Log Management Tool \fg
        \end{enumerate}
    \item Choix des outils à évaluer
        \begin{enumerate}
                \item Pour cette étape, je vais consulter plusieurs classements de système de gestion de logs et choisir ceux qui sont le plus souvent cités. Je vais probablement en prendre 5 ou 6.
        \end{enumerate}
    \item Synthèse et rédaction des slides
    \end{enumerate}
    Ce lundi, j'ai défini les critères d'évaluation, selon les demandes de DEPsys ainsi que les critères lu lors de mes recherches.
\subsection{Mercredi 11 mars}
    J'ai fais un tableau pour le choix des outils à tester. J'ai donc effectué un classement selon 4 tops de système de gestion de logs.
    J'ai également commencé l'évaluation à proprement parler, en particulier sur Elastic Stack et Graylog. J'ai également eu un problème de stockage de la base de donnée Zotero et j'ai perdu toute ma bibliographie.
\subsection{Jeudi 12 mars}
    J'ai continué l'évaluation avec Loggly, j'ai créé la présentation de synthèse pour la réunion avec Nastaran, puis je l'ai présentée.
\subsection{Mercredi 18 mars}
    J'ai remis en place Zotero, cette fois avec un synchronisation en ligne de ma bibliographie. J'ai analysé les différents systèmes de gestion de logs que j'hésitais à inclure dans l'évaluation. J'ai donc écarté ManageEngine EventLog Analyzer pour sa popularité vraiment faible et son manque de documentation, et PRTG Network Monitor, qui est très axé sur l'analyse d'un réseau, comme son nom l'indique. Je vais donc évaluer Splunk.
\subsection{Jeudi 19 mars}
    J'ai fais l'évaluation de Splunk. J'ai également eu la réunion hebdomadaire avec Nastaran.
\subsection{Samedi 21 mars}
    J'ai reformaté mon rapport avec le template \LaTeX écrit par Mateo Tutic. J'ai également développé la partie Choix des différents systèmes à évaluer.
\subsection{Dimanche 22 mars}
    J'ai continué la partie Choix des différents systèmes à évaluer. J'ai également télécharger la suite Elastic et testé avec les logs systèmes de Ubuntu. Cela fonctionne normalement.
\subsection{Lundi 23 mars}
    J'ai mis en place les systèmes de gestion de logs Elastic Stack, Graylog, Splunk et Loggly. J'ai testé (en insérant des logs et regardant le débit) les 3 premiers. Encore quelques problèmes pour Loggly (pour l'instant, il sauvegarde 1 log avec n lignes dans le message plutôt que n logs avec 1 ligne). J'ai également terminé les tableaux récapitulatif de l'évaluation de chaque système.
\subsection{Mardi 24 mars}
    J'ai terminé les tests d'ingestions de logs pour les 4 systèmes. J'ai commencé à faire mes slides pour la présentation du 25 mars.
\subsection{Mercredi 25 mars}
    J'ai terminé les slides de la présentation. J'ai fais la présentation du travail de Bachelor à l'entreprise DEPsys. S'en est suivi une discussion avec Pascal, Jonathan, Nastaran et moi au sujet de la suite de l'évaluation de mon TB, puis un ajustement du cas d'utilisation à implémenter fourni par Pascal.
\subsection{Jeudi 26 mars}
    J'ai commencé à refaire des tests d'ingestion de log. Cette fois-ci avec tout le pipeline. Je commence avec Elastic Suite, en y intégrant logstash afin qu'il filtre les données.
\subsection{Mercredi 01 avril}
    J'ai continué les tests d'ingestion avec Elastic Suite et rencontré beaucoup de problème. J'arrive à faire fonctionner un pipeline Logstash-Elasticsearch-Kibana, et un pipeline Filebeat-Elasticsearch-Kibana, mais pas un contenant les 4 logiciels de la suite.
\subsection{Jeudi 02 avril}
    J'ai continué les tests en tentant plusieurs tutoriaux trouvé sur internet. Mais je rencontre toujours des problèmes. Ils sont probablement liés à la communication entre Filebeat et Logstash. J'ai également suivi les tuto officiels de la Suite Elastic, mais ça n'a pas fonctionné non plus. Ceci est peut-être dû à mes fichiers de configurations des logiciels Filebeat et Logstash. J'ai ensuite eu une réunion avec Nastaran.
\subsection{Mercredi 08 avril}
    J'ai continué les tests d'ingestion. En suivant les guides du site d'elastic, j'ai remarqué qu'il y avait toute une section expliquant l'utilisation de la Suite avec Docker. Je me suis dit qu'il y avait plus de chance que cela fonctionne étant donné l'uniformité que propose Docker. Malheureusement, j'ai toujours les mêmes problèmes. Même en prenant un git public sensé fonctionner. Je me dit alors que le problème vient peut-être de mes fichiers de logs de tests (ils contiennent le même log multiplié n fois).
\subsection{Jeudi 09 avril}
    Je suis repassé sur une version non dockerisée de la suite Elastic. J'ai téléchargé un fichier de log d'un serveur Apache afin de tester la Suite avec un fichier de log réel, et fait d'autres modifications, notamment sur les fichiers de configuration (j'ai créé la partie \og filtrage \fg de Logstash avec un site internet permettant de créer ces filtres de manière incrémentale). Et ça à fonctionné. J'ai ensuite eu la réunion avec Nastaran.
\subsection{Mercredi 15 avril}
    

\end{document}
